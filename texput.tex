% Emacs, this is -*-latex-*-

\title{Information Theory}
\maketitle

\tableofcontents

\section{Rate}

In a communication system, the
\href{https://en.wikipedia.org/wiki/Bit_rate}{rate} describes the
amount of data (for example, bits) that is transmited by time unit
(for example, a second).

\section{Distortion}

In an lossy encoding system, the distortion (expressed for example by
the \href{https://en.wikipedia.org/wiki/Mean_squared_error}{MSE})
measures the amount of error between two signals: the original signal
and the distorted one. The origin of this error can be quite varied
and ranges from transmission errors to quantization processes.

\section{RD (Rate/Distortion) curve}

When the signal is quantized, rate and distortion are ``conflicting''
aspects in the sense that, for example, if the bit-rate is descreased,
the tue distortion is increased, and viceversa, usually. Such variables can be represented as a curve

Notice also that the DCT is orthonormal, and therefore, the matrix of
the forward transform is the transpose of the matrix of the backward
transform~\cite{sayood2017introduction}. This also means that the
contribution of the synthesis filters (inverse transform) to the
reconstructed signal are independent and have exactly the unity
gain\footnote{To find the gains (of any 1D transform) we can compute
the energy of the signal generated by the inverse transform of the
impulse discrete 1D signal
\begin{equation}
  \delta_{i}(x) = 
  \left\{
  \begin{array}{ll}
    1 & \text{if $i=x$}\\
    0 & \text{otherwise},
  \end{array}
  \right.
\end{equation}
where the
\href{https://en.wikipedia.org/wiki/Energy_(signal_processing)}{energy
  of a discrete signal} ${\mathbf s}$ is defined as
\begin{equation}
  \langle {\mathbf s}, {\mathbf s} \rangle =  \sum_{i}{{\mathbf s}_i^2}.
\end{equation}
}
%}}}

\section{Resources}
%{{{ 
\renewcommand{\addcontentsline}[3]{}% Remove functionality of \addcontentsline
\bibliography{data-compression,signal-processing}
%}}}
